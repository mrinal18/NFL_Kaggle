\documentclass{article}

% if you need to pass options to natbib, use, e.g.:
%     \PassOptionsToPackage{numbers, compress}{natbib}
% before loading neurips_2018

% ready for submission
% \usepackage{neurips_2018}

% to compile a preprint version, e.g., for submission to arXiv, add add the
% [preprint] option:
%     \usepackage[preprint]{neurips_2018}

% to compile a camera-ready version, add the [final] option, e.g.:
     \usepackage[final]{neurips_2018}

% to avoid loading the natbib package, add option nonatbib:
%     \usepackage[nonatbib]{neurips_2018}

\usepackage[utf8]{inputenc} % allow utf-8 input
\usepackage[T1]{fontenc}    % use 8-bit T1 fonts
\usepackage{hyperref}       % hyperlinks
\usepackage{url}            % simple URL typesetting
\usepackage{booktabs}       % professional-quality tables
\usepackage{amsfonts}       % blackboard math symbols
\usepackage{nicefrac}       % compact symbols for 1/2, etc.
\usepackage{microtype}      % microtypography
\usepackage{graphicx}
\graphicspath{ {./} }
\title{NFL Safety and Health : Helmet Safety}

% The \author macro works with any number of authors. There are two commands
% used to separate the names and addresses of multiple authors: \And and \AND.
%
% Using \And between authors leaves it to LaTeX to determine where to break the
% lines. Using \AND forces a line break at that point. So, if LaTeX puts 3 of 4
% authors names on the first line, and the last on the second line, try using
% \AND instead of \And before the third author name.

\author{%
  Mrinal, Archana, Sai, Chaithanya, Lauren
  \thanks{} \\
  Department of Computer Science\\
  Georgia State University\\
 Atlanta, GA 30302 \\ \\
 $$mmathur4@student.gsu.edu,\\
  abenkkallpallichand1@student.gsu.edu,\\
  vnuthalapati1@student.gsu.edu,\\
  sgaddam4@student.gsu.edu,\\
  ljames26@student.gsu.edu$$ 
  % examples of more authors
  % \And
  % Coauthor \\
  % Affiliation \\
  % Address \\
  % \texttt{email} \\
  % \AND
  % Coauthor \\
  % Affiliation \\
  % Address \\
  % \texttt{email} \\
  % \And
  % Coauthor \\
  % Affiliation \\
  % Address \\
  % \texttt{email} \\
  % \And
  % Coauthor \\
  % Affiliation \\
  % Address \\
  % \texttt{email} \\
}

\begin{document}
% \nipsfinalcopy is no longer used
\maketitle
\begin{abstract}
The National Football League (NFL) and Amazon Web Services (AWS) are teaming up to develop the best sports injury surveillance and mitigation program. In previous competitions, Kaggle has helped detect helmet impacts. As a next step, the NFL wants to assign specific players to each helmet, which would help accurately identify each player's “exposures” throughout a football play.We are trying to implement a computer vision based ML algorithms capable of assigning detected helmet impacts to correct players via tracking information.
\end{abstract}
\section{Introduction}
The National Football League is America's most popular sports league. Founded in 1920, the NFL developed the model for the successful modern sports league and is committed to advancing progress in the diagnosis, prevention, and treatment of sports-related injuries. Health and safety efforts include support for independent medical research and engineering advancements as well as a commitment to work to better protect players and make the game safer, including enhancements to medical protocols and improvements to how our game is taught and played.

\subsection{Problem Statement}
This problem proposes to identify the head injuries and mitigation problems in the field. This is for the safety of the player and the NFL currently does this manually by allotting a number to a player and mapping it.

\subsection{Approach/Solution}
Our approach is based on the classic detection within the video using object recognition or by treating each frame as images via which we can try to use image recognition to detect the impact. We are given many videos and labels which can be thoroughly looked upon and used for detection.
There are many algorithms which works well with it but we want to try and make something new and propose a method for faster and efficient detection by sampling the videos and randomly cropping them and train on single frame basis or we can try to generate a attention network which not only detects the frame in current videos but will also help in detecting in “future” videos.
Models we are trying to use: initially we will be testing it using MobileNet and YOLOv5 but then will try to build a model on top of it to increase the efficiency .
\section{Implementation}
\label{headings}

\subsection{EDA Analysis on Dataset }

Data consists of videos which contains of one sideline and one endzone view. They are aligned with frames between videos. Train set is with corresponding videos as well as labels in ${trian_labels.csv}$ , similarly, test set is with corresponding videos and labels in test/ directory 

They also provided additional dataset of images for better detection of helmet which has labeled bounding boxes and they are present in ${image_label.csv}$.
They provide a baseline for helmet detection boxes foe both training and testing set in ${train_baseline_helmets.csv}$ which was trained on images and labels in images folder
Size: 3.43 GB\\
Size of dataset: 9947,   \\    training images: 7957,      \\ validation images: 1990\\
Video shape: (160, 720, 1280, 3)\\
Data is raw and needs preprocessing and many more feature extractions! 

\subsubsection{Preprocessing}
we need to make sure the features which needs to be extracted are of proper shape and the input size is relevant to whatever distributions we will have
\subsubsection{Training and Testing Models}
The three models that we are consider are:
\begin{itemize}
\item Faster Region-based Convolutional Neural Networks (Faster R-CNNs),
\item You Only Look Once (YOLO),
\item Single Shot Detectors (SSDs).
\end{itemize}
\subsubsection{Baseline Models}
 We are considering You Only Look Once (YOLO) as the baseline as the number of the frames processed is highest when compared to Faster R-CNNs and SSDs.
We need to Train our model to make the detection of helmet collision and determine injured player based on helmet label assigned.
\subsubsection{Evaluation}
We will determine the best model that best fits based on comparison of different factors (Confusion matrix
Accuracy,
Precision,
Recall,
Specificity,
F1 score,
Precision-Recall or PR curve,
ROC (Receiver Operating Characteristics) curve
PR vs ROC curve.)from model testing results.
\section{Citations, figures, tables, references}
\label{others}

\subsection{Citations}

\subsection{Figures}
\includegraphics[scale=0.5]{Helemt vs frames Distribution}\\
\texttt{ Distribution of Helmets with respect to frames.}
\subsection{Tables}

\subsubsection*{Acknowledgments}

We thank Dr. Plis Sergey for his guidance, suggestions and lectures that paved a way for getting us started with this project.

\section*{References}

\medskip

\small
[1]https://www.kaggle.com/c/nfl-health-and-safety-helmet-assignment/

[2]https://towardsdatascience.com/finding-the-best-distribution-that-fits-your-data-using-pythons-fitter-library-319a5a0972e9

[3]https://www.youtube.com/watch?v=vIL-I3D-Zdw$\&ab_channel=Weights\%$26Biases

[4]https://arxiv.org/abs/2108.12711

[5]https://arxiv.org/abs/2109.00373

[6]https://arxiv.org/abs/2108.08339

[7]https://arxiv.org/abs/2108.11055

[8]https://arxiv.org/pdf/2108.12711.pdf


\end{document}